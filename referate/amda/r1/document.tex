\documentclass[12pt]{article}

\usepackage{url} % websites in bib
\usepackage{hyperref}
\usepackage[a4paper,left=1in,right=1in,top=1in,bottom=1in]{geometry} % margins
\usepackage{paralist} % compact enumerates

\begin{document}
	\title{K-Means clustering variants and improvements}
	\author{Stefan Sebastian, 242}
	\maketitle
	
	\begin{abstract}
		TODO make abstract
	\end{abstract}

	\newpage
	\tableofcontents
	\newpage
	
	\section{Introduction}
	\subsection{Motivation}
	In recent times there has been an explosive growth of data. This new, digital resource has become so important that a recent article from The Economist magazine has called it 'the new oil'\cite{TheMostValuableResource}. Some important sources of data, gathered by Bernard Marr\cite{HowMuchDataDoWeCreateEveryDay}, a business consultant in big data and AI technologies, are: Social Media (every minute 456000 tweets are sent on Twitter, 46740 photos are uploaded on Instagram, 4146600 YouTube videos are being watched), Communication (16 million text messages and 156 million emails are sent every minute), Services(45788 trips are made through Uber per minute), Internet of Things(8 million people use voice control every month). The amount of data in the world is predicted to reach 163 zettabytes by 2025\cite{WhatWillWeDoWhenTheWorldsDataHits163Zettabytes}, where one zettabyte represents a trillion gigabytes.
	
	The main reason for choosing the K-means method for this paper is because I believe that the development of a computationally efficient unsupervised learning method will be necessary to keep up with future's data, which will continue to grow and sometimes lack meaningful labels.
	
	\subsection{Clustering}
	Clustering is a field of data analysis whose purpose is to organize points, or objects into natural clusters. In other words given N objects described by a set of features, find a number of groups, such that objects in the same cluster are similar to each other and different from objects in other clusters\cite{DataClustering50yearsBeyondKmeans}. Given that there is no rigorous definition of similarity and that clusters in real data can overlap there are multiple algorithms and difference measures proposed in research.
	
	\subsection{K-means}
	The K-means algorithm was discovered independently in multiple scientific fields around the same time and its name comes from a paper published by MacQueen in 1967\cite{DataClustering50yearsBeyondKmeans}. Despite being so old, this method is still widely used in clustering tasks because of its efficiency, simplicity and proven success.
	
	The main goal of this algorithm is to minimize the distance between cluster centers and points assigned to them, which is a NP-hard problem. For this reason K-means is a greedy algorithm and has a chance to get stuck in local minimum instead of the best solution\cite{DataClustering50yearsBeyondKmeans}.
	
	A basic version of the K-means algorithm is\cite{AlgorithmsForClusteringData}:
	\begin{compactenum}
		\item Select an initial partition with K clusters and repeat 2 and 3 until the point membership no longer changes
		\item Assign each point to the closest center
		\item Update cluster centers based on the new assignment
	\end{compactenum}
	
	\section{Improvements}
	
	\subsection{Execution time optimizations}
	As illustrated in the introductory section, data sets are growing larger and more varied. Consequently there is a need for algorithms that scale well and obtain good results without having huge computational costs. K-means is such an algorithm, since it is characterized by simplicity and has been proven to work empirically with a multitude of datasets\cite{DataClustering50yearsBeyondKmeans}. This success has driven researchers to scale the algorithm for use on larger datasets. Next, I will present some of these execution time optimizations for the algorithm.
	
	\subsubsection{Enhanced K-means}
	A simple improvement for the basic K-means has been proposed by Fahim et al.\cite{EfficientEnhancedKmeans} under the name of "enhanced K-means". The main idea is that we can save some time during the computation of distances between points and cluster centers in the update phase of the original algorithm. This distance is calculated at every iteration of the algorithm but we could use the information from previous iterations, by using an additional data structure that memorizes the previous distance between a point and its cluster's center. At every iteration the distance to the new cluster mean is calculated and if it's less or equal to the previous distance stored for that point then we leave the point in its cluster and there's no need to calculate the other k-1 distances.
	
	The complexity of this algorithm can be approximated to O(nk) compared to the complexity of the classic algorithm which is O(nki), where n represents the number of points, k the number of clusters and i the maximum number of iterations. In brief, the time complexity when updating the position of a point is O(1) if it stays in its cluster and O(k) otherwise. Since the algorithm converges to a local minimum then the number of points updated in each iteration decreases, meaning the expected complexity is \( nk\sum_{j=1}^{i}1\mathbin{/}j \).
	
	The performance of the algorithm was tested on 3 different datasets, with the following number of records and features: (20000, 16), (4177, 7), (6574, 15) and was compared with the classic K-means and the CLARA algorithm. The enhanced K-means obtains similar results on smaller number of clusters (approximatively less than 50) but outperforms the other methods on larger values being about twice as fast as CLARA and 5 times as fast as the classic variant on more than 100 clusters. In conclusion this algorithm can be a useful optimization for working with datasets in which we expect to find a large number of clusters.
	
	\subsubsection{K-means clustering using MapReduce}
	MapReduce\cite{MapReduce} is a framework for processing large datasets over a cluster of computing nodes. It offers a programming model based on map and reduce functions and it has gained popularity in the big data community due to its simplicity, fault tolerance and scalability. This framework is a good starting point for clustering implementations that can work at large scale.
	
	Zhao et al.\cite{ParallelKmeansClusteringBasedOnMapReduce} provide a detailed description of a simple parallel K-means implementation using MapReduce. The algorithm, called 'PKMeans' uses only one MapReduce job: a map function that assigns each sample to the closest center and a reduce function which updates the cluster centers. The input dataset is stored on a distributed file system as a sequence of (key, value) pairs, where the key is the position of record in the data file, and the value is a string content of the record. Center values are stored on a global variable, so map functions can calculate distances independently. After each map task a combine function is applied which computes partial sums of the points assigned to the same cluster and their count. The reduce function takes as input the result of the combine task and uses it to compute the new means. The algorithm was tested with a number of computers varying between 1 and 4 and datasets of 1 up to 8GB and reported a linear speedup in computation time, meaning a system with m systems runs the algorithm m times faster. However, this speedup should decrease on larger systems due to increases in communication costs.
	
	
	\subsection{Seed selection}
	TODO seed selection stuff
	
	\section{Variants}
	\subsection{Bisecting K-means} 
	bisecting kmeans
	
	\subsection{Evolutionary K-means }
	genetic kmeans
	
	
	\section{Conclusions}
	TODO something conclusive
	
	\newpage
	\bibliography{references_document}
	\bibliographystyle{ieeetr}
\end{document}